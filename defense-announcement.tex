\documentclass{article}

\input{project-info}

\usepackage{graphicx}
\usepackage{amsmath}
\usepackage{float}
\usepackage{listings}
\usepackage{xcolor}
\usepackage[margin=0.7in]{geometry}
\usepackage{changepage}

\colorlet{grey}{gray!120}

\pagenumbering{gobble}

\begin{document}

\begin{center}

\includegraphics[width=0.3\textwidth]{media/ci-logo.png}\\

\hfill\break

\LARGE
\textbf{\color{grey}Computer Science Master Project Presentation}\\

\hfill\break
\hfill\break

\Large
{\bf \thesistitle}\\

\vspace{5mm}

\large
{\bf \Amogh Peechu}\\

\vspace{5mm}

\large
\textit{ Examination Committee:\\
{\bf \advisorname} (Advisor), {\bf Dr. Michael Soltys}}\\

\hfill\break

\end{center}

\begin{adjustwidth}{1in}{1in}
\textit{\bf Abstract:}\\

\vspace{3mm}

\normalsize
\noindent Phishing attacks, where malicious websites trick users into revealing sensitive information, are increasingly common and pose significant cybersecurity risks. This project focuses on creating a machine learning-based system to identify and classify phishing websites. The focus is on analyzing URL characteristics to identify potentially malicious sites. We propose a model that utilizes a variety of features extracted from URLs, including domain names, path structures, and query parameters. By training classifiers such as Random Forest, XGBoost, and Meta Model on a dataset of legitimate and phishing URLs, we aim to achieve high accuracy and robustness in phishing detection.
\end{adjustwidth}

\hfill\break

\begin{center}

\LARGE
{\bf 2pm-3pm, Thursday, Dec 12\textsuperscript{rd}, 2024\\
Sierra Hall 1111}\\

\vspace{15mm}

\large
{\bf All students and faculty are invited}\\

\hfill\break

\small
{\it An Academic Affairs Event}

\end{center}

\end{document}
